\documentclass[10pt,twoside]{tesisIER}

\usepackage{hhline}					% para dibujar lineas
\usepackage{fancyheadings}				% este lo necesita el estilo tesisIER.cls
\usepackage{blindtext}            % Dummy text

\usepackage{hyperref}   % Hipervinculos
\usepackage{pdflscape} %  para rotar páginas
\usepackage{longtable} % or {lscape}
\usepackage[activeacute,spanish]{babel}

%%	CAMBIO A MEXICANO

\addto\captionsspanish{\renewcommand{\contentsname}{Contenido}}
\addto\captionsspanish{\renewcommand{\listfigurename}{Lista de Figuras}}
\addto\captionsspanish{\renewcommand{\listtablename}{Lista de Tablas}}
\addto\captionsspanish{\renewcommand{\tablename}{Tabla}}


%%	HEADERS
\pagestyle{fancyplain}
\renewcommand{\chaptermark}[1]{\markboth{#1}{}}
\renewcommand{\sectionmark}[1]{\markright{\thesection\, #1}}
\lhead[\fancyplain{}{\bfseries\thepage}]%
      {\fancyplain{}{\bfseries\rightmark}}
\rhead[\fancyplain{\leftmark}{\bfseries\leftmark}]%
      {\fancyplain{}{\bfseries\thepage}}
\cfoot[]{}





%%	VECTORES
\newcommand{\vc}{\textbf c}
\newcommand{\BPSP}{\begin{pspicture}}
\newcommand{\EPSP}{\end{pspicture}}
\newcommand{\vF}{\textbf F}
\newcommand{\vg}{\textbf g}
\newcommand{\vG}{\textbf G}
\newcommand{\vn}{\textbf n}
\newcommand{\vN}{\textbf N}
\newcommand{\vPi}{\textbf \Pi}
\newcommand{\vq}{\textbf q}
\newcommand{\vr}{\textbf r}
\newcommand{\vu}{\textbf u}
\newcommand{\vv}{\textbf v}
\newcommand{\vU}{\textbf U}
\newcommand{\BE}{\begin{equation}}
\newcommand{\EE}{\end{equation}}

%%	VARIOS
\newcommand{\PROM}[1]{\left\langle #1\right\rangle}
\newcommand{\lsim}{\mathrel{\hbox{\rlap{\lower.55ex\hbox{$\sim$}} \kern-.3em 
 \raise.4ex \hbox{$<$}}}}




%%	PATH FIGURAS
\graphicspath{{figuras/}}  %define el subdirectorio donde latex busca las figuras, se pueden definir varios\usepackage{times}\renewcommand{\familydefault}{cmss}   % cambia el tipo de letra
%\usepackage{cite}                                      % se usa para la bibliograf'ia este no lo usa ulises
\usepackage[spanish,activeacute]{babel}				% este es el delos acentos as'i
\usepackage[utf8]{inputenc}				% este es el delos acentos normales así
\usepackage{amsmath}					% fonts m'as bonitas en las ecuaciones
%\usepackage{natbib}					% es el mero mero para la bibliografia
\usepackage{appendix}					% te permite usar ap'endices
\usepackage{amsfonts}					% m'as fonts
\usepackage{graphicx}				% include figures 
\usepackage{color}					% para definir colores
\usepackage{colortbl}					% m'as colores



%%	VARIOS

\title{Mi tesis en \LaTeX sin sufrir$^{\text{mucho}}$ en el proceso}
\author{Guillermo Barrios del Valle}

%\includeonly{resumen}
%\includeonly{introduccion}
%\includeonly{ebrv2}
%\includeonly{experimentosv2}
%\includeonly{conclusiones}
%\includeonly{validacion}



\begin{document} 
%
\spanishdecimal{.}
%
%\thispagestyle{empty}

\begin{pspicture}(13,19)
%\psgrid
\rput[C](1.1,17.3){\includegraphics[width=2.cm]{logo-unam.eps}}
\rput[C](1.1,.8){\includegraphics[width=2.cm]{logo-cie.eps}}
\psline[linewidth=0.2mm](2.2,17.2)(13,17.2)
\psline[linewidth=0.8mm](2.2,17.1)(13,17.1)
\rput[C](7.55,18.){ {\large\textbf{UNIVERSIDAD NACIONAL AUT\'ONOMA DE M\'EXICO}}}
\psline[linewidth=0.2mm](1,16)(1,1.4)
\psline[linewidth=0.8mm](0.8,16)(0.8,2)
\psline[linewidth=0.8mm](1.2,16)(1.2,2)

\rput[C](7.1,16.8){{\textbf{PROGRAMA DE MAESTR\'IA Y DOCTORADO EN}}}
\rput[C](7.1,16.3){{\textbf{INGENIER\'IA}}}


\rput[C](7.1,14.8){{{FACULTAD DE INGENIER\'IA}}}

\rput[C](7.1,12){{\textbf{LEVITACI\'ON DE PART\'ICULAS}}}
\rput[C](7.1,11.6){{\textbf{EN ONDAS DE SONIDO}}}
\rput[C](7.1,11.2){{\textbf{USANDO EL M\'ETODO DE LA}}}
\rput[C](7.1,10.8){{\textbf{ECUACI\'ON DE BOLTZMANN EN REDES }}}



\rput[C](7.1,9){{\textbf{T E S I S}}}

\rput[C](7.1,8){{QUE PARA OPTAR POR EL GRADO DE:}}

\rput[C](7.1,7){{\textbf{DOCTOR EN INGENIER\'IA}}}
\rput[C](7.1,6.5){{\textbf{\'AREA  MEC\'ANICA}}}
\rput[C](7.1,6){{\textbf{OPCI\'ON TERMOFLUIDOS}}}

\rput[C](7.1,5){ {P R E S E N T A : }}
\rput[C](7.1,4.5){{\textbf{M. I. GUILLERMO BARRIOS DEL VALLE}}}

\rput[C](7.1,3.5){ {T U T O R : }}
\rput[C](7.1,3){{\textbf{DR. RA\'UL RECHTMAN SCHRENZEL}}}

\rput[C](7.1,2){ {AGOSTO 2007 }}

\end{pspicture}

\newpage
\thispagestyle{empty}


{\textbf JURADO ASIGNADO:}
\vspace{1cm}

Presidente: Dr. Jaime Cervantes de Gortari
\vspace{0.5cm}

Secretario: Dr. Jorge Antonio Rojas Men'endez
\vspace{0.5cm}

1er Vocal:  Dr. Ra'ul Mauricio Rechtman Schrenzel
\vspace{0.5cm}

2do Vocal:  Dr. H'ector Lorenzo Ju'arez Valencia
\vspace{0.5cm}

3er Vocal:  Dr. Francisco Javier Solorio Ordaz
\vspace{1cm}

Centro de Investigaci'on en Energ'ia, Temixco, Morelos, M'exico.
\vspace{1cm}

{\textbf TUTOR DE TESIS:}
\vspace{1cm}

Dr. Ra'ul Rechtman Schrenzel










% 
%
\frontmatter
%
\chapter*{}
\thispagestyle{empty}


\begin{center}
A la comunidad  de software libre
\end{center}

\newpage
\thispagestyle{empty}
\chapter*{Agradecimientos}

Definitivamente este documento no hubiera sido posible sin  Raúl Rechtman (mi asesor durante mi maestría y doctorado, y ahora un gran amigo) quién me puso tres reglas al iniciar a trabajar con él: Programar en C, escribir en \LaTeX y graficar en Gnuplot.
%
\tableofcontents
%
\listoffigures

%
\mainmatter
\chapter{Introducción}
\label{chap:introduccion}

 
 
 \section{Introducción}


Este documento reune mi experiencia en \LaTeX desde el 2001 conocí Linux y el software libre. Además, espero se enriquezca con las peticiones que se hagan al autor por medio de GitHub, correo electrónico.

En el Capítulo~\ref{chap:escritura} veremos como escribir secciones, subsecciones,  referencias a capítulos o secciones, citas con bibtex, ecuaciones, listas y tablas.

En el Capítulo~\ref{chap:figuras} será sobre las diferentes formas de incluir esquemas y figuras.

Dado que ahora muchos se han mudado a Overleaf, en el Apéndice~\ref{chap:overleaf} encontrarán las recomendaciones para trabajar en esa plataforma.


Recuerda revisar el código fuente para ver  como se hacen las cosas, descarga, comparte, comenta y te invito a hacer una contribución a este proyecto.

\chapter{Literature review}
\label{chap:review}



\section{Psychrometric aspects}
\begin{itemize}
	\item Ideal gases
	\item Mixed gases
	\item Psychrometric aspects
			\begin{itemize}
				\item Air-vapour mix
				\item Dalton law
				\item Humidity ratio
				\item Relative humidity
				\item Enthalpy of atmospheric air
				\item Psychrometric chart and 							different temperatures.
			\end{itemize}
		
\section{Human comfort and air conditioning}

\section{Evaporative cooling}

\section{Energy plus}
\end{itemize}
\chapter{Methodology}
\label{chap:methodology}

 
 
 \section{Project description}
 
 \begin{itemize}
  	\item Papiit
 	\item Temixco
 	\item Grafica de radiación
 	\item Hay potencial
 	\item cafetería modeling
 	\item aspersores, direct evaporative modelling, foto del osm
 \end{itemize}
 

This thesis work is part of the Papiit project, Estudio teórico-experimental del enfriamiento evaporativo en eficicaciones.
Objetivo de Papiit. 

The experiments were carried out with data of Temixco. Temixco is a city located in the mexican state of Morelos, it has a latitud of 18.85°, longitud of -99.22° and 1253 MSL. According to the population and housing census made in 2020 by the Instituto Nacional de Estadística, Geografía e Informática (INEGI)[1]the city has a population of 122,263 people. 
Weather



 
 
 \section{Numerical experiments}
 
 Hay que esperar un poco, pero podría ser numerical simulation and validation… pero ya que tengamos más información lo consideramos.


También hay que considerar si habrá algunos apéndices, reportando tus libretas, me parece interesante documentar tu proceso de aprendizaje.
 
 \section{Validation process}
 

\chapter{Results}
\label{chap:results}

 
 
\chapter{Conclusions}
\label{chap:conclusions}

 


%
%
\appendix
%
%
\backmatter
\bibliographystyle{unsrt} %unsrt,alpha,abbrv,plain
\bibliography{bibliografia}
%
%
\end{document}
