\chapter{Methodology}
\label{chap:methodology}

 
 
 \section{Project description}
 
 \begin{itemize}
  	\item Papiit
 	\item Temixco
 	\item Grafica de radiación
 	\item Hay potencial
 	\item cafetería modeling
 	\item aspersores, direct evaporative modelling, foto del osm
 \end{itemize}
 

This thesis work is part of the Papiit project, Estudio teórico-experimental del enfriamiento evaporativo en eficicaciones.
Objetivo de Papiit. 

The experiments were carried out with data of Temixco. Temixco is a city located in the mexican state of Morelos, it has a latitud of 18.85°, longitud of -99.22° and 1253 MSL. According to the population and housing census made in 2020 by the Instituto Nacional de Estadística, Geografía e Informática (INEGI)[1]the city has a population of 122,263 people. 
Weather



 
 
 \section{Numerical experiments}
 
 Hay que esperar un poco, pero podría ser numerical simulation and validation… pero ya que tengamos más información lo consideramos.


También hay que considerar si habrá algunos apéndices, reportando tus libretas, me parece interesante documentar tu proceso de aprendizaje.
 
 \section{Validation process}
 
